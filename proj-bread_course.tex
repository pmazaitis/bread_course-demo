% Project file for demonstration course materials.
%
% This file is part of a suite...
%

% Our project starts with a declaration in this file between two pieces of markup: a \startproject command and a \stopproject command. The \startproject command takes a single argument: a label for the project.

\startproject bread_course-demo

% Environment Files
%
% Using ConTeXt's project structure, we can load document setups for
% every document in the project, here at the project level. This is
% very useful for keeping the design of all the documents consistant!

% We do not need to include the extension with the \environment
% command, but it doesn't hurt anything if we do.

% I find it useful to seperate out different functionality into
% smaller, forcued environment files. The -debug file can be uncommented
% to increase debugging output, but on the logs and on the page.

% \environment env-bread_course-debug
\environment env-bread_course-general
\environment env-bread_course-typography
\environment env-bread_course-layout
\environment env-bread_course-forms

% Now we need to reference our products. For this demo, each product 
% is a semester's worth of course materials: syllabus, schedule, 
% assignment prompts, etc. 

% I have invented the ficticious course id 'BRD101' here.

\product BRD101-2018-01 % Spring semester, 2018

% In future, we can spin up new semesters by adding new products:

%\product BRD101-2018-03
%\product BRD101-2019-01
%\product BRD101-2019-03

% ...and populating those project areas with materials, as needed.


\stopproject
